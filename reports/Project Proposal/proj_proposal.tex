\documentclass[11pt, a4paper, twoside, openright]{report}
\usepackage{float} % lets you have non-floating floats
\usepackage{url} % for typesetting urls
%  We don't want figures to float so we define
\newfloat{fig}{thp}{lof}[chapter]
\floatname{fig}{Figure}

\title{Genetic Programming for Antarctic Ice Sheet Modelling}
\author{Samuel Mata}
\usepackage[font, ecs]{vuwproject} 
\supervisors{Dr. Bach Nguyen, Dr. Bing Xue}
\otherdegree{Bachelor of Science with Honors in Artificial Intelligence}
\date{}

\begin{document}

% Make the page numbering roman, until after the contents, etc.
\frontmatter

\begin{abstract}
  With the growing need for accurate long-term modelling of
  the Antarctic Ice Sheets, the currently used statistical
  models do not provide the necessary computational 
  efficiency for sufficiently long term predictions. This
  project aims to investigate and evaluate the application
  of Genetic Programming \textit{(GP)} and Evolutionary
  Learning techniques as a potential alternative to 
  traditional methods.
\end{abstract}


\maketitle
\mainmatter


\section*{1. Problem Statement}

The changing conditions of Antarctica's Ice Sheets are a 
significant factor in global climate change, particularly
with respect to rising sea levels. In this context, the
ability to accurately model the long-term behaviour of
these ice sheets is of great importance. Current approaches
are not computationally efficient enough to provide
meaningfully long-term predictions within practical time
limits. This project aims to investigate the application of
Genetic Programming \textit{(GP)} and  Evolutionary Learning
techniques as a potential alternative. There is some
challenge in the application of these techniques, as the
models produced must be accurate and efficient enough to
meaningfully improve the practicality of long-term
predictions. Furthermore, for the results to be trusted
the models must be explainable and interpretable.


\section*{2. Motivations}

Current methods of modelling Antarctic Ice Sheet measurements 
- such as those being undertaken at Victoria University's Antarctic 
Research Center \textit{(ARC)} - are typically based on
traditional statistical methods, which are computationally
intensive and time consuming. This inefficiency limits the
practical scope of predictions, excluding the potential of
longer-term forecasts. The use of machine learning models
shows promise as an approach due to the improved 
computational efficiency of these predictive systems. 
Specifically, the use of Genetic Programming \textit{(GP)} 
and Evolutionary Learning techniques provide the 
greatest potential due to the increased explainability
these methods provide allowing for greater understanding in
the results provided. This is a novel application for
these techniques, and so the potential for improvement
compared to traditional methods is worth investigation.

\section*{3. Goals}

The primary goal of this project is to investigate the
application of Genetic Programming and Evolutionary Learning
techniques on the problem of Antarctic Ice Sheet modelling.
This will involve several stages:
\begin{enumerate}
\item \textbf{An initial Exploratory Data Analysis \textit{(EDA)}} of
  the data provided by the ARC, to greater understand the
  the nature of the problem and the data. \textit{( 2 weeks)}
  
  This will involve the use of univariate, bivariate, 
  and multi-variate analysis to identify the distribution,
  shape, and relationships of the features.

  Variables should also be analysed in terms of their 
  spatial and temporal distributions.
  
  Finally, potential feature engineering opportunities
  should be explored to identify potential improvements.
\item \textbf{Several iterations of model development} and evaluation
  using Genetic Programming and Evolutionary Learning
  techniques, to identify the most effective approach.
  \textit{(~8 weeks)}

  Each model should explore implementing a different
  approach, before analysing and evaluating the results
  against previous iterations.
\item \textbf{Further development of the most effective approach},
  targetting the explainability and interpretability of
  the predicted outputs. \textit{(~4 weeks)}

  This will involve analysing and visualising the
  underlying mathematical expressions evolved by the model,
  and checking these against current scientific
  understanding of the problem.
\item \textbf{A final evaluation of the most effective model}, to
  determine the potential for improvement over traditional
  methods. \textit{(~4 weeks)}

  Evaluation should consider many measures, particularly the
  computational efficiency of the model, the accuracy of the
  predictions, and the interpretability of the results.
\end{enumerate}
Project milestones should be set for each of these stages
to ensure that the project remains on track. It is important
to note the potential for these goals to change as the 
direction of the project becomes clearer.

\section*{4. Evaluation}

The success of the project will be evaluated on three main
criteria:
\begin{enumerate}
\item \textbf{The accuracy of the the model's predictions.}
  
  It is important that the developed model is able to
  provide accurate predictions of the Antarctic Ice Sheet
  measurements, and that these predictions closely follow
  the measurements provided by ARC.
\item \textbf{The computational efficiency of the model.}

  Any developed model should maintain a sufficiently high
  computational efficiency to allow for long-term predictions
  without exceeding practical time limits.
\item \textbf{The interpretability of the model's results.}

  To verify and validate the results of the model, the 
  predicted results should also be available for 
  interpretation and explanation as to how those results
  were reached.
\end{enumerate}
These criteria can be used as metrics to objectively
evaluate the performance of the models developed, and
to compare them against traditional methods.

\section*{5. Resource Requirements}

No external or additional resources are required for the
project as all tooling is publicly and freely accessible.

%%%%%%%%%%%%%%%%%%%%%%%%%%%%%%%%%%%%%%%%%%%%%%%%%%%%%%%
\backmatter
%%%%%%%%%%%%%%%%%%%%%%%%%%%%%%%%%%%%%%%%%%%%%%%%%%%%%%%

%\bibliographystyle{ieeetr}
\bibliographystyle{acm}
\bibliography{sample}
\end{document}
